
\documentclass[preprint]{sigplanconf}
\usepackage[latin1]{inputenc}
\usepackage{times}
\usepackage{latexsym}
\usepackage{graphicx}
\usepackage{enumitem}
\usepackage{listings}
\usepackage{bussproofs}
\usepackage{comment}
\usepackage[numbers]{natbib}
\usepackage{graphicx} % Graphics support
\usepackage{caption} % Improved captions
\usepackage{subfig} % Inclusion of subfigures


\usepackage{tikz}
\usepackage{tikz-qtree}

\usepackage{algorithm}
\usepackage{algpseudocode}

% Set the overall layout of the tree
%\tikzstyle{level 1}=[level distance=3.5cm, sibling distance=3.5cm]
%\tikzstyle{level 2}=[level distance=3.5cm, sibling distance=2cm]

% Define styles for bags and leafs
%\tikzstyle{leaf} = [draw=none,fill=none, text width=4em, text centered]
%\tikzstyle{relation} = [ellipse, draw, text width=5em, text centered]
%\tikzstyle{branch} = [rectangle, draw, text width=6em, text centered]




%\usepackage{subfigure}

% adjust space between letters
\usepackage{microtype}

% adjust space between lines
\linespread{1.05}

% for header and footer
\usepackage{fancyhdr}

% change the font size of section
%\usepackage{sectsty}
%\sectionfont{\fontsize{12}{15}\selectfont}
%\subsectionfont{\fontsize{11}{14}\selectfont}

\lstset{morekeywords={in_range, in_rel, add, remove, Drop, Deliver, ForwardTo, Multicast, IpSrc, IpDst}}

\usepackage{amsmath, amssymb}

\pagestyle{fancy}

\lhead{Department of Computer Science\\
Princeton University}
\chead{}
% \rhead{\includegraphics[height=15mm]{princeton_logo.jpg}}
\lfoot{}
\cfoot{}
\rfoot{}

% remove page numbers
%\thispagestyle{empty}

% no indent at the start of paragraphs
\parindent=0em

% for increasing space between paragraphs
\newcommand{\emptyline}{\vspace{5pt}}

\title{\textit{Red Oak}\\ Proactive Controller Generation for Algorithmic Policy}
 \authorinfo{Ryan Beckett}{Princeton University}{rbeckett@princeton.edu}
 \authorinfo{Olivier Savary B\'{e}langer}{Princeton University}{olivierb@princeton.edu}


%-----------------------------------------------------------

\begin{document}
\maketitle


\begin{abstract}
Software Defined Networking (SDN) is a paradigm that gives programmers 
centralized visibility and control over the network. However, due to performance 
demands, writing correct and efficient programs is challenging. Programmers must 
reason about how to correctly forward packets in the network, how to maintain
global network state, which packets to observe at the centralized controller, 
and how to minimize latency by pushing as much decision logic
as possible onto the high-performance switch elements.

We developed a high-level SDN library that can be embedded into existing languages.
Our library exposes an API that allows programmers to directly express behavioral intent, 
while leaving the performance details to the library's optimization algorithms.
Our library supports basic tests on packet fields as well as a simple mutable table
data structure for storing global network information (e.g., firewall ACL, MAC learning table).
Finally, to ensure adequate performance, 
we statically analyze the programmer's policy by pushing symbolic packets through the policy 
and observing the resulting packet forwarding decision.
\end{abstract}


\section*{Introduction}

One of the major outcomes of the Software Defined Networking (SDN) movement has been a unified, open interface to the switch hardware. This has, in turn, enabled new opportunities for network programmability. The typical model for SDN involves a centralized controller or distributed set of controllers programming switches dynamically to implement the desired control functionality.
However, the question remains as to how best to write programs for SDNs.

Recent collaborations between researchers in networking and programming languages have resulting in a number of new network-specific programming languages. One example is the Frenetic family of languages \cite{Frenetic}, which allow programmers to write simple packet processing functions in a domain-specific language and combine them together using policy combinators. Another approach, described in Maple \cite{Maple}, is to allow programmers to write policies in an arbitrary, high-level programming language as a function that takes a packet and produces a forwarding path, the algorithmic policy. Conceptually, the function is run anew on each packet that enters the network.

Ultimately, the performance requirements of networks dictate that the centralized controller(s) should not handle most traffic since redirecting packets to be handled by the controller incurs orders of magnitude increased latency compared to those being handled by switches. For this reason, compiler technologies for network programming languages must be able to push much of the control plane functionality into the data plane switches themselves. 

In Maple \cite{Maple}, the authors show how to optimize the controller to reduce the number of packets are handled by running the function on them. Maple does so by providing a packet API capturing each access to packet fields, and memoizing the results of the function for subsequent packets sharing the accessed fields' values. Unfortunately, there are a number of limitations to this approach:


\begin{enumerate}
\item Since controller decisions are memoized at runtime, all packets must initially go to the controller before it can learn how to forward similar packets directly in the data plane. This leads to a cold-startup phase with high packet latency.
\item The controller only observes concrete packets. This can lead to installing overly specialized forwarding rules on the switches that miss some of the high-level intent of the programmer (e.g., matching all packets with a particular IP prefix).
\item Most interesting network functions keep some global state (e.g., MAC learning table, ACLs). To ensure memoized decisions do not become invalid when the controller's environment (global state) changes, programmers must manually decide which previous decisions to invalidate.
\end{enumerate}


In this work, we devised a language-independent API together with a compilation strategy that, like Maple, allows programmers to write network programs with the full generality of a high-level, general-purpose programming language.
However, we address each of the shortcomings  listed above. Our contributions are the following:

\begin{enumerate}
\item We describe a static analysis on the user-defined forwarding policy that explore all possible paths of execution by evaluating the function on symbolic packets. This analysis is then used to proactively generate a network controller and initial forwarding rules to install in the network.
\item We provide a packet API that lets programmers ask if a field is in a particular range (e.g. an IP prefix) rather than if the field equals a particular value. This more expressive test works together with our analysis to better capture programmer intent when setting up forwarding decision in the data plane.
\item We provide a mutable relation data structure in our API to maintain global state. The data structure works together with our analysis to provide an automatic invalidation scheme. The analysis determines which future packets might affect global state and ensures that these relevant packets are actually seen by the controller.
\end{enumerate}



\section*{Algorithic Policies Revisited}
% Explains what algorithmic policies are, how maple was handling them and the difference with our model 
Algorithmic policies were introduced by Maple as a model for SDN controllers. An algorithmic policy is a programmer-defined function from a packet to a forwarding decision (i.e., where the packet(s) should end up). Conceptually this model offers the abstraction that the network forwards packets as if this function were executed on each packet when entering the network. To handle future traffic directly in the data plane, Maple keeps a tree data structure that branches on decisions observed at run time (e.g., the packet had source IP 10.0.0.1)


A policy for a simple stateful firewall using our API is included in Figure~\ref{fig:firewallcode}. For simplicity, we assume that IP ranges are between 0 and 1000, however it is easy to generalize this to actual IP prefixes.
In the example, source IPs from 0 to 10 are considered to be internal to the network and everything else, external. 

If a packet is sent from an internal host to an external host, the pair of internal and external host is added to a list of trusted connection. If a packet is received from an external host, we lookup in the table to see if the connection is trusted, in which case we forward the packet to its destination, otherwise we drop it. The forwarding decision "Deliver" is just shorthand for "forward the packet according to its intended destination".
An interesting point to be made about this function is that we are actively manipulating global policy state when we add entries to the stateful firewall's trusted connection relation.


\begin{figure}[ht]
\begin{lstlisting}
f(pkt):
  if in_range(pkt, src, 0, 10) then
     if in_range pkt dst  [(0,10)] then
        Deliver
     else 
        add pkt (IpDst, IpSrc) ``conn'';
        Deliver
  else if in_rel pkt (IpSrc,IpDst) ``conn'' then
         Deliver
       else
         Drop
       end
  end                
  \end{lstlisting}

\caption{A Stateful Firewall}
\label{fig:firewallcode}
\end{figure}

From an algorithmic policy such as the one included in Fig~\ref{fig:firewallcode}, we generate a decision tree summarizing the function's interaction with the packet and the relevant decisions that will be made for particular classes of packets. The decision tree is created by analyzing the programmers function statically and then used by our runtime system to install switch rules and update them only when necessary (i.e., in response to changes in global state). For the stateful firewall included in figure~\ref{fig:firewallcode}, the following decision tree would be generated:


\begin{figure}[ht]
\includegraphics[scale=.5]{img/dtree.png}
\caption{Decision Tree for Stateful Firewall}     
\label{fig:decisiontree}  
  \end{figure}

This decision tree would, in turn, lead to the following set of prioritized symbolic rules. We call these symbolic rules because they represent constraints on the actual concrete rules. The concrete rules our system will install in the network depend on the current contents of the trusted connection relation. In particular, for each IP source that is trusted to communicate with an internal destination in the "conn" relation, we will get a concrete rule permitting this communication.
  \begin{lstlisting}[mathescape]
src $\in$ [0, 100] $\mapsto$ Controller
src $\in$ [101, 1000], (src,dst) $\in$ conn  $\mapsto$ Deliver
src $\in$ [101, 1000], (src,dst) $\notin$ conn $\mapsto$ Drop
  \end{lstlisting}

After initialization (starting with no trusted connections), the concrete rules would be generated from the symbolic rules by having all relations be empty:
\begin{lstlisting}[mathescape]
src $\in$ [0, 100] $\mapsto$ Controller  
src $\in$ [101, 1000] $\mapsto$ Drop  
\end{lstlisting}
  

  When a packet is received, it is forwarded according to the first matching concrete rule installed on the switch. If the packet is sent to the controller, we evaluate the decision tree for that packet, potentially adding and removing tuples from relations, before forwarding the packet according to forwarding decision held in the relevant leaf of the decision tree. We then regenerate the concrete rules using the updated relation. 
  
  For example, a packing coming from \lstinline|src| 150 and \lstinline|dst| 10 would be matched by the second rule, and dropped on the switch.
Subsequently, on a packet with \lstinline|src| 10 and \lstinline|dst| 150, the first rules would be matched, evaluating the decision tree would add tuple $(10,150)$ to relation \lstinline|conn|, the packet would be forwarded according to the forwarding decision \lstinline|Deliver|, and the concrete rules would updated as follow:

\begin{lstlisting}[mathescape]
src $\in$ [0, 100] $\mapsto$ Controller  
src = 150, dst = 10 $\mapsto$ Deliver  
src $\in$ [101, 1000] $\mapsto$ Drop  
\end{lstlisting}
  
In this example, packet traveling from \lstinline|150| to \lstinline|10| would be matched by the second rule on the switch and delivered, while packets from \lstinline|150| to other internal hosts or from other external host to any internal hosts would be dropped.

\section*{Algorithmic Policies and Decision Trees}
In this section, we describe how an algorithmic policy may interact with a packet and with relations to generate a forwarding decision, and how we build a decision tree out of an algorithmic policy.

   \subsection*{Range-based Branching}
   

   % Why range instead of actual value, pre-generation of rules using symbolic pre-execution
   Our API exports a single function for inspecting the contents of a packet's field, which is a test on the range of the packet's field. The ability to test if a value falls in a particular range is useful for capturing programmer intent with respect to things like IP prefix matching. Other tests on packet fields can be encoded in terms of ranges. For example, to test whether a field equals a particular value, one can simply test if it falls in the range with that value as both the lower and upper bound. Likewise, the programmer can test if a field is greater than or less than a particular value by testing if the field is in the range from the value to the maximum or minimum possible value for that field respectively. 
   
Conveniently, ranges on packet fields also give us a way to describe symbolic packets - packets that represent a set of concrete packets. Later we will see how our static analysis will proactively generate rules to install on the switch hardware by propagating symbolic packet ranges through the control structure of the program.

   % Can represent constraints on the ranges admissible given a certain topology as a series of ``true'' branches at the root of the tree.


     \subsection*{Built-in Relation with Automatic Invalidation}
   % Relation with add and remove tuple of fields from a relation (empty at the begining), Inhabitant-based Branching 
	Many network applications of interest need to keep around some global state to keep track of things that have happened in the network. A stateful firewall, like the one we introduced earlier, needs to maintain an ACL of external hosts that are trusted to communicate with internal hosts. This list can change over time as packets traverse the network. Likewise, a MAC learning application would need to remember where to send packets destined for particular Ethernet destinations. However, reasoning about which packets the controller needs to see, and how to set up forwarding rules in the data plane to ensure that these packets are actually seen by the controller is challenging for programmers.
	
	We introduce a simple table (relation) data structure that handles updating mutable, global state in the controller automatically. Intuitively, each time a table can be updated, the controller must witness the update. Our range-based analysis determines all possible packets that could reach code resulting in an update to the table and instruments the data plane with forwarding rules to send relevant traffic to the controller.

   
   % Computing which packets need to be sent to the controller (i.e. only the ones reaching an ``add'' or a ``remove'' node.

     
  

   \subsection*{Forwarding Decision}
   % can give constant decision (e.g. drop() or fwd(loc) to a certain location loc), or decision based on values of the packet (e.g. deliver() which forwards to dst)
	The programming model for Oak is that each packet entering the network is mapped to a forwarding decision that describes where the packet should end up leaving the network. For simplicity, we assume that we are targeting a one big switch implementation of the network. The one big switch model has been heavily optimized \cite{Obs}, and we view the problem of determining paths through the network as orthogonal to our work here. 
  
   We provide several types of forwarding decisions to the programmer. The programmer can deliver a packet, according to the packet's desination, drop a packet, and forwarding a packet to a static location. Our model also allows for dynamic forwarding based on component of relation, for example to the last seen location of a host according to a MAC-learning controller (not shown in Fig~\ref{fig:range_api}). Finally, the programmer can combine several forwarding decisions into one by using a mulitcast forwarding decision.
   

   \subsection*{Combining Algorithmic Policies}
   % cross, ``holes''
   As algorithmic policies under our model are functions of the host language, we inherit the latter's function space ability, allowing, amongst other things, for compositon of algorithmic policies and higher-order algorithmic policies.

\begin{figure}[ht]
\begin{lstlisting}

let filter_src_range f r pkt =
 if in_range pkt IpSrc r then
    f pkt
 else
    Drop   
\end{lstlisting}

\caption{Example of a Partial Algorithmic Policy}
\label{fig:ex-hole}
\end{figure}

   
   For example, we include in Figure~\ref{fig:ex-hole} the source code for \lstinline|filter_src_range| a partial policy, which, given a policy \lstinline|f| and a range \lstinline|r|, will forwards all packets whose source IP is within range \lstinline|r| and drop the packet otherwise. The ability to compose two or more policies in sequence like this arises many times in practice. In our earlier firewall example, instead of delivering packets that were not dropped, we could instead have deferred to another function to decide how to handle trusted traffic.


   \begin{figure}[ht]
\begin{lstlisting}
let f_cross (f1) (f2) pkt =
  let d1 = f1 pkt in
  let d2 = f2 pkt in
  Multicast (d1,d2)
\end{lstlisting}

\caption{Example of a Composition Operator}
\label{fig:ex-cross}
\end{figure}
   
Various composition operator may be built as host language function to combine multiple algorithmic policies into a single one. For example, \lstinline|f_cross|, included in Figure~\ref{fig:ex-cross}, runs algorithmic policy \lstinline|f_1| and \lstinline|f_2| on input packets and multicasts them according to both forwarding decisions. This corresponds to the parallel composition operator for the Frenetic language \cite{Frenetic}, however here the composition takes place in a fully general purpose language.
   
\section*{Implementation}
%that's the hope...

\subsection*{A.P.I.}
In our model, algorithmic policies interact with packets through A.P.I function calls before returning a forwarding decision. We include in Figure~\ref{fig:range_api} the core of our \lstinline|A.P.I.|.

\begin{figure}[ht]
  \begin{lstlisting}[mathescape]
type pkt
type field = IpSrc | IpDst 
type fields = field list
type range = (int*int) list

type decision =
	| Deliver
	| Drop
	| ForwardTo of int
	| Multicast of decision * decision

val min_value: int 
val max_value: int

val in_range: pkt $\rightarrow$ field $\rightarrow$ range $\rightarrow$ bool
\end{lstlisting}

\caption{Signature of the packet and range-branching portion of the API}
\label{fig:range_api}
\end{figure}


\lstinline|field| enumerates the different fields of a packet. It can be extended to represent more fields of a packet's header. Associated with \lstinline|field| are \lstinline|range|s, possible values taken by a certain packet's field, within a given minimum value \lstinline|min_value| and maximum value \lstinline|max_value|. \lstinline|range| consists of list of integer pairs, representing a collection of intervals whose union correspond to the range of values of a packet. For simplicity, we consider integer fields here, but this could be extended, for example, to IP prefix, boolean flags, etc.

\lstinline|forwarding_decision| enumerates the different forwarding decision which can be returned by an algorithmic policy. \lstinline|Deliver| forwards the packet to its destination, \lstinline|Drop| drops the packet, \lstinline|ForwardTo i| forwards the packet to a constant location \lstinline|i| and \lstinline|Multicast fd1 fd2| duplicates the packet and handles each according to \lstinline|fd1| and \lstinline|fd2| respectively.

Algorithmic policy may base their forwarding decision on the result of calling the \lstinline|in_range pkt fld r|, which returns whether value of packet \lstinline|pkt| at field \lstinline|fld| is in the range \lstinline|r|.

\begin{figure}[ht]
  \begin{lstlisting}[mathescape]
type relation = string

val add: pkt $\rightarrow$ fields $\rightarrow$ relation $\rightarrow$ unit
val remove: pkt $\rightarrow$ fields $\rightarrow$ relation $\rightarrow$ unit
val in_rel: pkt $\rightarrow$ fields $\rightarrow$ relation $\rightarrow$ bool
\end{lstlisting}

\caption{Signature of the relation portion of the API}
\label{fig:rel_api}
\end{figure}


In Figure~\ref{fig:rel_api}, we include the relation portion of the \lstinline|A.P.I.|, used to model stateful controller programs. A \lstinline|relation| represents a table in which each row contains a series of field values. It is referred to by a string. Policy writer may add a row to a relation \lstinline|rel| by calling the function \lstinline|add pkt flds rel|, which will populate the new row with the fields \lstinline|flds|' values of the current packet \lstinline|pkt|. Similarly, they may remove a row containing exactly the values of fields \lstinline|flds|' values of the current packet \lstinline|pkt| by calling \lstinline|remove pkt flds rel|. Finally, one may test if the values of fields \lstinline|flds| of current packet \lstinline|pkt| are in relation \lstinline|rel| by calling \lstinline|in_rel pkt flds rel|.


\begin{figure}[ht]
  \begin{lstlisting}[mathescape]
type examples =  ((field * int) list) list
  
val compile: (pkt $\rightarrow$ decision) $\rightarrow$ policy
val run: policy $\rightarrow$ examples $\rightarrow$ unit
  \end{lstlisting}

  \caption{Signature of the interactive portion of the API}
  \label{fig:build_api}
\end{figure}


The rest of the API, included in Figure~\ref{fig:build_api}, consists of functions used to interact with Red Oak. \lstinline|compile| is used to trace a policy from a given function from packet to forwarding \lstinline|decision|. \lstinline|run| simulates the compiled \lstinline|policy| on a series of \lstinline|examples|. In the remainder of this section, we describe these functions in more detail.

\subsection*{Internal Representation of Packets and Policies}

Packet are represented abstractly in the API as type \lstinline|pkt|, and are treated as concrete packet by algorithmic policies. In contrast, internally, \lstinline|pkt| represents a class of packet which could reach a certain point in the program. We call this class of packet a symbolic packet.
A symbolic packet consists of two parts:
\begin{itemize}
\item a map from \lstinline|field| to \lstinline|range|, representing the range of possible value from the class of packet
  \item a list of pairs of \lstinline|relation| and list of \lstinline|field| representing the relation that the relevant \lstinline|field|'s value of the symbolic packet should be in. 
  \end{itemize}
  
A \lstinline|policy| is represented internally as a decision tree whose leaves are forwarding decision and whose node are either branching according the range of a field value, according to a relation, or adding and removing components from a relation. To simplify rule generation, we also record in the tree the most general symbolic packet who can reach leaves, \lstinline|Add| and \lstinline|Remove| nodes.


\subsection*{Tracing the Algorithmic Policy}
From an algorithmic policy, we generate a decision tree using the function \lstinline|compile|.  

\begin{figure*}[ht]
\begin{lstlisting}[mathescape]
let compile (f: pkt $\rightarrow$ forwarding_decision) : policy =
      push most general packet to (empty) stack
      set the current location to be the root of an empty decision tree
      while (stack is not empty)
         pop the top of the stack as pkt
         run f of pkt resulting in a forwarding decision fd and a refined packet pkt
         record the forwarding decision for the symbolic packet pkt in the decision tree
         set the current location to be the root of the decision tree and repeat
  
\end{lstlisting}
\caption{Pseudocode for the compile function}
\label{fig:compile-pseudo}
\end{figure*}

To do so, we repeatedly run the algorithmic policy \lstinline|f| on symbol packets generated to explore every reachable path.
We begin by calling \lstinline|f| on a symbolic packet covering the full range of value for each of its fields.

When \lstinline|f| calls \lstinline|in_range| to learn if the field \lstinline|fld|'s value is in the range \lstinline|r|, we
\begin{enumerate}
\item  create an \lstinline|Inrange| branching node in the decision tree
\item refine the current symbolic packet to represent all packets in range \lstinline|r| that could reach this test in the algorithmic policy and return true
\item push on the stack of packets to process a symbolic packets representing all packets in the complement of \lstinline|r| for field \lstinline|fld|.
\end{enumerate}


Similarly, when the algorithmic policy calls \lstinline|in_relation| to learn if fields \lstinline|flds|'s value are in a relation \lstinline|rel|, we
\begin{enumerate}
  \item create an \lstinline|Inrelation| branching node in the decision tree
  \item tag the current packet as being in the relation, return true
  \item push on the stack of packets to process the same packet tagged as not being part of relation \lstinline|rel|.
\end{enumerate}
When \lstinline|f| calls \lstinline|add| or \lstinline|remove|, we record in the decision the current symbolic packet, the relation and the fields being added or removed from the relation before continuing with the execution of \lstinline|f|.


% Prune dead subtree on empty ranges and add/remove which do not affect any inhabitant-based branching.



When \lstinline|f| returns with a forwarding decision, we record it, paired with the current symbolic packet, as a leaf node at the current position in the decision tree. We then pop the next packet from the stack of packets to process and restart tracing the policy from the root of the decision tree, exploring another path of \lstinline|f|.  

\subsection*{Generating the Symbolic and Concrete Rules}

When \lstinline|compile| terminates, a complete decision tree has been traced for the input algoritmic policy. The next step is generate symbolic rules, mapping a symbolic packet to a forwarding decision.
To do so, we do a depth-first, left to right traversal of the tree. On reaching an \lstinline|Add| or a \lstinline|Remove| node, we create a new rule sending the relevant packets to the controller, pruning the rest of the sub-tree. On reaching a leaf, we create a new rule forwarding relevant packets according to the forwarding decision held in the leaf node.

We then generate concrete rules from a collection of relations and a list of symbolic rules. Concrete rules differ from symbolic rule in that they do not depend on relations. Each symbolic rule expands into a list of concrete rules. For each of the tuple in the relation on which the rule depend, we create a new rule, refining the ranges of the relevant fields of the symbolic packet with the values of the tuple. We repeat this operation on the created list of symbolic rules until all dependant relations are taken care off, at which point the rules are considered concrete. Due to the left-to-right order of collection, we do not need to modify symbolic rules marked as not in a certain relation, as all the packets in the relation will have been matched by the positive rule before reaching this one.

For example, taking the following symbolic rules: 
  \begin{lstlisting}[mathescape]
src $\in$ [101, 1000], (src,dst) $\in$ conn $\mapsto$ Deliver
src $\in$ [101, 1000], (src,dst) $\notin$ conn $\mapsto$ Drop
  \end{lstlisting}
  and the relation ``conn'' as \lstinline| {(150,10), (200, 9), (8, 150)}|, we start by expanding the first rule to:
\begin{lstlisting}[mathescape]
src $\in$ ([101, 1000] $\cap$ 150), dst = 10 $\mapsto$ Deliver
src $\in$ ([101, 1000] $\cap$ 200), dst = 9 $\mapsto$ Deliver
src $\in$ ([101, 1000] $\cap$ 8), dst = 150 $\mapsto$ Deliver
 \end{lstlisting}
which reduces to:
\begin{lstlisting}[mathescape]
src = 150, dst = 10 $\mapsto$ Deliver
src = 200, dst = 9 $\mapsto$ Deliver
 \end{lstlisting}
Note the omission of the third concrete rule, resulting from \lstinline|ipsrc| having an empty range. The second symbolic rule is then taken directly as concrete rule, with the packet in the relation having been matched by the previously generated concrete rule, resulting in:
\begin{lstlisting}[mathescape]
src = 150, dst = 10 $\mapsto$ Deliver
src = 200, dst = 9 $\mapsto$ Deliver
src $\in$ [101, 1000] $\mapsto$ Drop
\end{lstlisting}

\subsection*{Simulating the Network Controller}

We provide a function \lstinline|run| to simulate the controller policy on a sequence of packets. \lstinline|policy| is an abstract datatype solely returned by \lstinline|compile| (see Fig.~\ref{fig:compile-pseudo}). In addition to the policy, \lstinline|run| takes in a list of packets represented as list of pairs of \lstinline|field| and \lstinline|int|, representing the values of the different fields of the packet. We expect all fields to be set exactly once for each packet in the list.

\lstinline|run| starts by building the symbolic rules for the given policy, and generating the concrete rules assuming all relations to be empty. In sequence, packet are matched with the current concrete rules, recording the forwarding decision. If a packet is sent to the controller, we evaluate the policy on the packet, potentially adding or removing tuples from relations. We then re-generate the concrete rules with the updated relation, before recording the forwarding decision.

\begin{figure*}[ht]
\begin{lstlisting}
let run (pol: policy) (inputs: examples) : unit =
      build the symbolic rules for policy pol
      generate the concrete rules with all relations empty
      for each packets in input,
          find the first rule matching the packet, and record the forwarding decision fd
          if packet is sent to the controller then
              evaluate the policy on the packet, potentially updating relations
              regenerate concrete rules
              forward the packet according the evaluated forwarding decision
          else
              forward the packet according to fd 
\end{lstlisting}

\caption{Pseudocode for the run function}
\label{fig:run-pseudo}
  \end{figure*}


While we currently start the simulation with empty relation, \lstinline|run| could easily be extended to initialize the controller with pre-populated relations. 


\subsection*{Optimizations}
   % 1) expand tree before add/remove nodes with a match on that relation ( based on add(p); add(p) == add(p) so can shortcircuit if p is already in the relation)
The rules generated by the presented technique ensures that the only packets being forwarded to the controller are those reaching an add or a remove node and thus potentially modify the state of the controller. However, these packets do not always modify the state of the controller. For example, in the firewall example (see Fig~\ref{fig:firewallcode}), after seeing a first packet with an \lstinline|ipsrc| of 10 and \lstinline|ipdst| of 150, any subsequent packet with the same \lstinline|ipsrc| and \lstinline|ipdst| could be delivered without being first forwarded to the controller, as adding the same tuple to a relation is idempotent. By using both local and global transformations on the decision tree, we can get closer to an optimal amount of packet forwarded to the controller, where only packets modifying relations are seen by the controller.

We implemented in \textit{Red Oak} a local transformation adding an \lstinline|in_rel| node before \lstinline|add| or \lstinline|remove| nodes. In the case of \lstinline|add|, the \lstinline|add| node is omited in the true branch of \lstinline|in_rel|, such that packets whose fields are already in the relation will not be seen by the controller due to this \lstinline|add| node. Similarly, we omit the \lstinline|remove| node in the false branch of \lstinline|in_rel|. In the firewall example (see Fig.~\ref{fig:firewallcode}), this would result the following decision tree:

\begin{figure}[ht]
  \includegraphics[scale=.33]{img/optdtree.png}
\caption{Optimized Decision Tree for Stateful Firewall}     
\label{fig:decisiontreeopt}  
  \end{figure}


From which the following symbolic rule would be generate:
\begin{lstlisting}[mathescape]
src $\in$ [0, 100], (dst, src) in conn $\mapsto$ Deliver
src $\in$ [0, 100] $\mapsto$ Controller
src $\in$ [101, 1000], (src,dst) $\in$ conn $\mapsto$ Deliver
src $\in$ [101, 1000], (src,dst) $\notin$ conn $\mapsto$ Drop
\end{lstlisting}

With empty relation, the concrete rules would be the same as with the unoptimized decision tree. However, after seeing, for example, packet with \lstinline|ipsrc| 10 and \lstinline|ipdst| 150, the rules would be
\begin{lstlisting}[mathescape]
src = 10, dst = 150 $\mapsto$ Deliver  
src $\in$ [0, 100] $\mapsto$ Controller  
src = 150, dst = 10 $\mapsto$ Deliver  
src $\in$ [101, 1000] $\mapsto$ Drop  
\end{lstlisting}


Other optimizations may further reduce the number of packet sent to the controller. For example, we could remove \lstinline|add| node which are followed by \lstinline|remove| node for the same relation and tuple on all path without the change being observed by an \lstinline|in_rel| node. Similarly, we could detect and warn the user if an added tuple may never be observed by an \lstinline|in_rel| node, for example by switching \lstinline|ipsrc| and \lstinline|ipdst| in the add tuple of the stateful firewall example (see Figure~\ref{fig:firewallcode}). We leave the implementation of these optimizations as future work.


% probably no evaluation :(
% \section*{Evaluation}
% 1) optimality w.r.t. packets seen by the controler?
% 2) more examples
% 3) simulated run


\section*{Related Work}
% deeper comparison with Maple

%???

\section*{Conclusion}
In this paper, we have introduced \textit{Red Oak}, a SDN library for proactively generating a dynamic controller from an arbitrary programmer-defined policy in a general purpose programming language.
\textit{Red Oak} provides programmers the abstraction that their centralized policy will run on each 
new packet entering the network, while alleviating for the programmer need to consider
how to split the decision logic between to switches and the controller.

The \textit{Red Oak} API exposes simple tests on packet fields as well as a global, mutable table data
structure for recording information about previous packets observed in the network.
To ensure that most packets are handled directly on the switches, we employ a static analysis
that pushes symbolic packets through the programmer's policy to explore how all classes of 
packets will be handled, determining which packets must be observed by the controller and which 
packets need not be. Finally, we describe our \textit{Red Oak} prototype's run-time system that 
specializes decisions after making concrete observations in the network that dictate forwarding policy.


\bibliographystyle{plainnat}
\bibliography{bibi}



\end{document}
















